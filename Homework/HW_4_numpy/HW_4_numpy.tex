\documentclass[11pt,oneside]{article}	

      % LOAD PACKAGES ==========================================================
	\usepackage{graphicx}
	\usepackage[utf8]{inputenc}
	\usepackage[american]{babel}
	\usepackage{amssymb}
	\usepackage[intlimits]{amsmath}
	\usepackage{array}
	\usepackage{mdwlist}
	\usepackage{subfig}		% Allows subfigs/subfloats
	\usepackage{algorithmic}
	\usepackage{lscape}
	\usepackage{rotating}		% Allows \begin{sideways} \end{sideways} for vertical table headers.	
	\usepackage{threeparttable}	% Allow footnotes in tables.	
	\usepackage{multirow}		% Allow table cells to span multiple rows/cols.
	\usepackage{hyperref}		% Allow \url{} and \href{url}{name}
	\usepackage{verbatim}
	\usepackage{enumerate}		% http://www.tex.ac.uk/cgi-bin/texfaq2html?label=enumerate
	\usepackage{color}		% Allow colored fonts
	\usepackage[toc,page]{appendix}
	\usepackage{tikz}
	\usepackage{lastpage} 		% \pageref{LastPage} = total number of pages.
	\usepackage{ifthen}		
	\usepackage{setspace} 		% Allows \singlespacing, \onehalfspacing, \doublespacing (set below)
	\usepackage{listings}
      % ========================================================================	

      % DEFINE PAGE FORMATTING +++++++++++++++++++++++++++++++++++++++++++++++++
      	% Select Line Spacing:
		 \singlespacing
		% \onehalfspacing		
		% \doublespacing	
	
	% Margins:
		\usepackage[letterpaper,left=1.0in,top=1.0in,right=1.0in,bottom=1.0in]{geometry}

	% Page Style
	  	\pagestyle{plain}	% Includes page number
		% \pagestyle{empty}	% Completely blank
      % ++++++++++++++++++++++++++++++++++++++++++++++++++++++++++++++++++++++++

      % MISC ITEMS +++++++++++++++++++++++++++++++++++++++++++++++++++++++++++++
      % By default all math is set to inline mode. The \displaystyle command
      % ensures that we don't get small fractions or summations with limits
      % on the sides.
		\everymath{\displaystyle}	
      % ++++++++++++++++++++++++++++++++++++++++++++++++++++++++++++++++++++++++


      % SETUP TikZ +++++++++++++++++++++++++++++++++++++++++++++++++++++++++++++
	\usetikzlibrary{arrows,shapes,matrix}
	\usetikzlibrary{decorations.pathmorphing} 
	\usepgflibrary{plotmarks}
	\usetikzlibrary{patterns}   
	\tikzstyle{block}=[draw opacity=0.7,line width=1.4cm]
      % ++++++++++++++++++++++++++++++++++++++++++++++++++++++++++++++++++++++++

	% This command creates a box marked ``Result'' around text.
	% To use type \result{8 em}.
	\newcommand{\result}[1]{\vspace{5 mm}\par \noindent
	\marginpar{\textsc{Result}} $\qquad\qquad$
	\framebox{\begin{minipage}[c]{0.75 \textwidth}
	\tt\begin{center} \vspace{#1} \end{center}\end{minipage}}\vspace{5 mm}\par}


      % DOCUMENT INFO ++++++++++++++++++++++++++++++++++++++++++++++++++++++++++
	  	\newcommand{\showKey}{0} 	% 1 --> Show answers, 0 --> Hide answers
      % ++++++++++++++++++++++++++++++++++++++++++++++++++++++++++++++++++++++++

\setlength{\parskip}{\baselineskip}%
\setlength{\parindent}{0pt}%

% ---------------------------DOCUMENT STARTS HERE---------------------------
\begin{document}

% ---------Begin Header-----------
\begin{center}
	\vskip-2em
	\begin{Large}
	IE 555 -- Programming for Analytics\\
	\end{Large}
	
	\bigskip
	Homework \#4 -- Working with NumPy\\

	\bigskip
	
	{\color{red}
		Due Date:  Thursday, March 30
	}

	\ifthenelse{\equal{\showKey}{1}}{%
		{\color{red}
		\textbf{SOLUTION KEY}
		}
	}%
	{
		% Nothing to do here.
	}% if not set to 1

\end{center}
% ---------End Title-----------

\hrule

\medskip

% --------------------------------	

\lstset{language=Python}          % Set your language (you can change the language for each code-block optionally)


%\begin{comment}
	\definecolor{mygreen}{rgb}{0,0.6,0}
	\definecolor{mygray}{rgb}{0.5,0.5,0.5}
	\definecolor{mymauve}{rgb}{0.58,0,0.82}

	\lstset{ %
	  backgroundcolor=\color{gray!10!white},   % choose the background color; you must add \usepackage{color} or \usepackage{xcolor}; should come as last argument
	  basicstyle=\ttfamily,        % the size of the fonts that are used for the code
	  breakatwhitespace=false,         % sets if automatic breaks should only happen at whitespace
	  breaklines=true,                 % sets automatic line breaking
	  captionpos=t,                    % sets the caption-position to bottom
	  commentstyle=\color{green!30!black},    % comment style
	  deletekeywords={...},            % if you want to delete keywords from the given language
	  escapeinside={\%*}{*)},          % if you want to add LaTeX within your code
	  extendedchars=true,              % lets you use non-ASCII characters; for 8-bits encodings only, does not work with UTF-8
	  frame=single,	                   % adds a frame around the code
	  keepspaces=true,                 % keeps spaces in text, useful for keeping indentation of code (possibly needs columns=flexible)
	  keywordstyle=\color{blue},       % keyword style
	  language=Python,                 % the language of the code
	  morekeywords={*,...},           % if you want to add more keywords to the set
	  numbers=left,                    % where to put the line-numbers; possible values are (none, left, right)
	  numbersep=5pt,                   % how far the line-numbers are from the code
	  numberstyle=\tiny\color{mygray}, % the style that is used for the line-numbers
	  rulecolor=\color{black},         % if not set, the frame-color may be changed on line-breaks within not-black text (e.g. comments (green here))
	  showspaces=false,                % show spaces everywhere adding particular underscores; it overrides 'showstringspaces'
	  showstringspaces=false,          % underline spaces within strings only
	  showtabs=false,                  % show tabs within strings adding particular underscores
	  stepnumber=1,                    % the step between two line-numbers. If it's 1, each line will be numbered
	  stringstyle=\color{mymauve},     % string literal style
	  tabsize=4,	                   % sets default tabsize to 2 spaces
	  title=\lstname,                   % show the filename of files included with \lstinputlisting; also try caption instead of title
	  xleftmargin=35pt,
	  xrightmargin=5pt, 
	  aboveskip=0pt,
	  belowskip=10pt
	}
%\end{comment}

Sometimes source data aren't in a format that we would've designed.  The purpose of this assignment is to help you practice using numpy and to work with poorly formatted data.

\section*{Assignment Details}

You need to write a function that calculates some statistics about student grades in a particular course.

\begin{itemize*}
	\item Your function should be in a Python script named \texttt{UCASEUBUSERNAME\_grades.py}; replace \texttt{UCASEUBUSERNAME} with your UB user name in all caps.
	
	\item Within this python script, you should write a function named \texttt{gradeInfo}.
	
	\item The \texttt{gradeInfo()} function will be called as follows: \\
		\texttt{gradeInfo(filename, numExams, hwWeight)}
		
		\begin{itemize*}
			\item See \texttt{UCASEUBUSERNAME\_grades.py} for information about these three input parameters.
		\end{itemize*}		
\end{itemize*}

Your \texttt{gradeInfo()} function should do the following:

\begin{enumerate}

\item Import the given .csv file.  See \texttt{grades\_example.csv}, which describes the structure of input files.

\item Return the following five (5) pieces of information, \textbf{in this order}:

	\begin{enumerate}
		\item Find the average of HW1. \\
			Return this as a scalar value in the range [0, 100].

		\item Sort the grades in descending order for HW2 (best grades first). \\
			Return as a ($n$ x 2) numpy array.  There are $n$ rows, where each row is a student.  The first column returned should be the student ID, the second column is the score on HW2 (as a score in the range [0, 100]).
	
		\item Find the students who made 90 or above on both HWs 1 and 3. \\
			Return as a 1-dimensional numpy array, just containing ID numbers.

		\item Find the number of students who made 80 or below on HW1 and 90 or above on HW2. \\
			Return as a scalar integer.
	
		\item Each homework is equally weighted.  Find each student's current average grade, rounded to 1 decimal place, in the range [0, 100]. \\
			Return as a ($n$ x 2) numpy array.  There are $n$ rows in the source data, where each row is a unique student.  The first column to be returned is the student ID, the second column is the weighted score in the range [0, 100].
	
			\begin{itemize*}
			 \item For example, suppose a student had the following scores: \\
				Homework:  9/10, 4/5, 35/50.  Exam1:	85/100 \\
		
				If hwWeight = 0.4, the student's average grade is
				( ( (9/10 + 4/5 + 35/50)/3 ) * 0.4 + ( (85/100)/1 ) * (1 - 0.4) ) * 100 = 83.0
			\end{itemize*}
	
	\end{enumerate}
\end{enumerate}
		

% - Plot a histogram of the scores for Exam 1.  Bin by <60, [60,70), [70,80), [80,90), >=90 
% 	Return the bins instead?
	


\end{document}



	



	
	